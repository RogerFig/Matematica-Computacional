\begin{Gabarito}{1}
~
        \begin{multicols}{4}
        \begin{enumerate}[a)]
            \item Finito
            \item Infinito
            \item Finito
            \item Infinito
        \end{enumerate}
    \end{multicols}

    
\end{Gabarito}
\begin{Gabarito}{2}
~
    \begin{multicols}{2}
    \begin{enumerate}[a)]
        \item $A = \{ 4, 5, 6, 7\}$
        \item $B = \{ Abril, Junho, Setembro, Novembro \}$
        \item $C = \{Bras\acute{i}lia\}$
        \item $D = \{ 0, 1, 8\}$
        \item $E = \{0, 1, 2, ...\}$
        \item $F = \{ 0 \}$
        \item $A = \{ 5, 6, 7, ...\}$
        \item $B = \{ 3, 4, 5\}$
    \end{enumerate}
\end{multicols}

\end{Gabarito}
\begin{Gabarito}{3}
~

    \begin{enumerate}[a)]
        \item

        \begin{itemize}
                \item $2 \in A$
                \item Se $n \in A$, então $n^2 \in A$
            \end{itemize}

        \item

            \begin{itemize}
                    \item $a_1 = 1$
                    \item $n \in \mathbb{N+}, a_{n+1} = a_{n} + 2n - 1$
                \end{itemize}

        \item

        \begin{itemize}
                \item $1 \in C$
                \item Se $n \in C$, então $3n \in C$
            \end{itemize}
    \end{enumerate}
\end{Gabarito}
\begin{Gabarito}{4}
~

    \begin{multicols}{4}
        \begin{enumerate}[a)]
            \item V
            \item V
            \item F
            \item V
            \item V
            \item F
            \item F (Operador $\in$ é aplicado a elementos e não a conjuntos)
            \item V
            \item V
            \item F (observar operador)
            \item F
            \item V
        \end{enumerate}
      \end{multicols}
\end{Gabarito}
\begin{Gabarito}{5}
~

    Seja $x \in A$. Então $x \in \mathbb{R}$ e $x^2 - 4x + 3 = 0$ ou $(x - 1)(x - 3) = 0$, o que nos dá $x = 1$ ou $x = 3$. Em qualquer dos casos, $x \in \mathbb{N}$ e $1 \leq x \leq 4$, de modo que $x \in B$. Portanto, $A \subseteq B$. O número 4 pertence a B, mas não pertence a A, logo $A \subset B$.
\end{Gabarito}
\begin{Gabarito}{6}
~

    Sejam $A = \{x | x \in x^2 < 15\}$ e $B = {x | x \in \mathbb{N} e 2x < 7}$.

    Para provar que $A = B$, vamos mostrar que $A \subseteq B$ e $B \subseteq A$. Para $A \subseteq B$, precisamos escolher um elemento arbitrário de A — ou seja, qualquer coisa que satisfaça a propriedade que caracteriza os elementos de A — e mostrar que satisfaz a propriedade que caracteriza os elementos de B. Seja $x \in A$. Então $x$ é um inteiro não negativo que satisfaz a desigualdade $x^2 < 15$. Os inteiros não negativos cujos quadrados são menores do que 15 são 0, 1, 2 e 3, logo esses são os elementos de A. O dobro de cada um desses inteiros não negativos é um número menor do que 7. Portanto, todo elemento de A pertence a B e $A \subseteq B$.

    Vamos mostrar agora que $B \subseteq A$. Todo elemento de B é um inteiro não negativo cujo dobro é menor do que 7. Esses números são 0, 1, 2 e 3, e cada um deles tem o quadrado menor do que 15, logo $B \subseteq A$.
\end{Gabarito}
\begin{Gabarito}{7}
~

    $\wp(A) = \{\emptyset, \{1\}, \{2\}, \{3\}, \{1, 2\}, \{1, 3\}, \{2, 3\}, \{1, 2, 3\}\}$.
\end{Gabarito}
\begin{Gabarito}{8}
~

    $2^n$ elementos.
\end{Gabarito}
\begin{Gabarito}{9}
~

    c
\end{Gabarito}
\begin{Gabarito}{10}
~

    \begin{multicols}{2}
    \begin{enumerate}[a)]
        \item - (anulada) [$A \cup B = \mathbb{N}$]
        \item F
        \item V
        \item - (anulada) [$A \cup C = A$]
        \item V
    \end{enumerate}
\end{multicols}
\end{Gabarito}
\begin{Gabarito}{11}
~

    \begin{multicols}{2}
    \begin{enumerate}[a)]
        \item 10
        \item $\{1,2,3,4,5,7,8,9,10\}$
        \item $\{ 1,2,3 \}$
        \item $\{ 2,8 \}$
        \item $\{ 1,2,3,4,6,7,9 \}$
    \end{enumerate}
\end{multicols}
\end{Gabarito}
\begin{Gabarito}{12}
~

    \begin{multicols}{2}
        \begin{itemize}
            \item $[(A \cup B) \cap C] \cup \left[(A \cup B) \cap \overline{C} \right]$
            \item $(A \cup B) \cap (C \cup \overline{C})$
            \item $(A \cup B) \cap S$
            \item $(A \cup B)$
        \end{itemize}

        \begin{itemize}
            \item (comutatividade)
            \item (distributividade)
            \item (complemento)
            \item (elemento neutro)
        \end{itemize}
    \end{multicols}
\end{Gabarito}
\begin{Gabarito}{13}
~

    $[C \cup (A \cap B)] \cap \left[(A \cap B) \cup \overline{C}\right] = A \cap B$

\end{Gabarito}
\begin{Gabarito}{14}
~

    \begin{multicols}{2}
        \begin{itemize}
            \item $A \cup (B \cap \overline{B})$
            \item $A \cup \emptyset$
            \item $A$
        \end{itemize}

        \begin{itemize}
            \item (distributividade)
            \item (complemento)
            \item (elemento neutro)
        \end{itemize}
    \end{multicols}
\end{Gabarito}
\begin{Gabarito}{15}
~
    \begin{enumerate}[a)]
        \item $\{ -1, 1 \}$
        \item $\{ -3, -2, -1, 0, 1, 2, 3 \}$
        \item $\{{4,8,12,16,20,...} \}$
    \end{enumerate}

\end{Gabarito}
\begin{Gabarito}{16}
~

    \begin{enumerate}[a)]
        \item $\{ x | (\exists y)( y \in \{1,2,3,4\} ~ e ~ x = y^2) \}$
        \item $\{ x | \text{x são os números primos}\}$
        \item $\{{x| \text{x é um inteiro não negativo escrito em forma binária}}\}$
    \end{enumerate}

\end{Gabarito}
\begin{Gabarito}{17}
~

    \begin{center}
        \includegraphics[width=.4\linewidth]{Figs/conjuntos-cropped.pdf}
    \end{center}

\end{Gabarito}
\begin{Gabarito}{18}
~

    \begin{center}
        \includegraphics[width=.4\linewidth]{Figs/conjuntos2.png}
    \end{center}
\end{Gabarito}
\begin{Gabarito}{19}
~

    \begin{multicols}{2}
    \begin{enumerate}[a)]
        \item $\{1,2,3,4,5\}$
        \item $\{1,3\}$
        \item $\{5,7,9,11,...\}$
        \item $\{0,5,6,7,8,9,11,13,15,17,...\}$
        \item $\{ 4, 2, \infty\}$
    \end{enumerate}
\end{multicols}

\end{Gabarito}
\begin{Gabarito}{20}
~

    \begin{enumerate}[a)]
        \item $\{(a,x),(a,y),(b,x),(b,y),(c,x),(c,y)\}$
        \item $\{(0,a),(0,b),(0,c),(1,a),(1,b),(1,c)\}$
    \end{enumerate}
\end{Gabarito}
\begin{Gabarito}{21}
~

    \begin{enumerate}[a)]
    \item
    \begin{multicols}{2}
        \begin{itemize}
            \item $A \cup (B \cap \overline{B})$
            \item $A \cup \emptyset$
            \item $A$
        \end{itemize}

        \begin{itemize}
            \item (distributividade)
            \item (complemento)
            \item (elemento neutro)
        \end{itemize}
    \end{multicols}

    \item

    Correção: $A \cap (B \cup \overline{A}) = B \cap A$

    \begin{multicols}{2}
        \begin{itemize}
            \item $(A \cap B) \cup (A \cap \overline{A})$
            \item $(A \cap B) \cup \emptyset$
            \item $(A \cap B)$
            \item $(B \cap A)$
        \end{itemize}

        \begin{itemize}
            \item (distributividade)
            \item (complemento)
            \item (elemento neutro)
            \item (comutatividade)
        \end{itemize}
    \end{multicols}
\end{enumerate}
\end{Gabarito}
\begin{Gabarito}{22}
~

    $A=\{1,3,5,6,7,8,9\}$ e $B=\{2,3,6,9,10\}$
\end{Gabarito}
\begin{Gabarito}{23}
~

    $X=\{1,3,5\}$
\end{Gabarito}
\begin{Gabarito}{24}
~


    a
\end{Gabarito}
