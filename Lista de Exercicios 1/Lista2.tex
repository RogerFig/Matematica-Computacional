\documentclass[a4paper, 12pt, addpoints]{exam}
\usepackage{IfbaListas}
%==============================================================
\NumList{2} %NÚMERO DA LISTA
\Assunto{Lógica de Predicados}
%==============================================================
%=========================================================
%------------Cabeçalho e rodapé 
%=========================================================
\pagestyle{headandfoot}%{head}%empty
\firstpageheader{}{}{}
\runningheader{\disciplina}{}{Lista \numlist: \assunto}
\runningheadrule
% \copyright \professor
\firstpagefooter{}{}{Pag. \thepage\ de \numpages}
%\iflastpage{Outros templates em \href{http://gg.gg/profwaldexsantos}{gg.gg/profwaldexsantos}
\runningfooter{}{}{Pag. \thepage\ de \numpages}
\runningfootrule
%===============================================================
%INFORMAÇÕES SOBRE A AVALIAÇÃO
\nomeInst{Instituto Federal da Piauí}
\logoInst{\includegraphics[scale=0.17]{Figs/IFPIPicosVertical.png}}
\Campus{Picos} %PARA ENSINO FUNDAMENTAL OU MÉDIO-COMENTE ESTA LINHA COM %
\nomeCurso{Análise e Desenvolvimento de Sistemas} %Se não for curso superior, basta comentar esta linha ou deixar em branco
\nomeDisciplina{Matemática Computacional}
\Semestre{1}
\nomeProfessor{Rogerio Figueredo de Sousa}
\Aluno{}
\matriculaAluno{}
%===============================================================
 %CARREGA AS INFORMAÇÕES GERAIS DO PROFESSOR, ESCOLA ETC
%==============================================================

%COMEÇO DO DOCUMENTO
\begin{document}
\info\vspace{-1 cm} %Imprime as informações do cabeçalho programada no pacote 
% inicia a gravacao da resposta no arquivo "ans" cujo nome externo é gabarito 
\Opensolutionfile{ans}[Gabarito]
%*****************************************************************************
\begin{questions}%COMEÇA O AMBIENTE DE QUESTÕES
  \question Determine o valor lógico de cada uma das fbf's. Suponha o conjunto universo todos números reais.
  \begin{multicols}{2}
    \begin{enumerate}[a)]
      \item $(\forall x)(|x| = x)$
      \item $(\exists x)(x^2 = x)$
      \item $(\exists x)(|x| = 0)$
      \item $(\exists x)(x + 2 = x)$
      \item $(\forall x)(x + 1 > x)$
      \item $(\forall x)(x^2 = x)$
      \item $(\exists x)(2x = x)$
      \item $(\exists x)(x^2 + 3x = -2)$
      \item $(\exists x)(x^2 + 5 = 2x)$
      \item $(\forall x)(2x + 3x = 5x)$
    \end{enumerate}
\end{multicols}

  \begin{resp}~
    \begin{multicols}{4}
          \begin{enumerate}[a)]
            \item i e iii
            \item ii
            \item iv
          \end{enumerate}
        \end{multicols}
  \end{resp}

  \question  Determine o valor lógico de cada uma das fbf's. Suponha o conjunto universo \{2,3,4,5,6,7,8,9\}.
  \begin{multicols}{2}
    \begin{enumerate}[a)]
      \item $(\forall x)(\forall y)(x + 5 < y + 12)$
      \item $(\forall x)(\exists y)(x \cdot y \text{ não é primo})$
      \item $(\exists y)(\forall x)(x \cdot y \text{ não é primo})$
      \item $(\exists x)(\exists y)(x^2 > y)$
      \item $(\forall x)(\exists y)(x^2 > y)$
      \item $(\exists x)(\forall y)(x^2 > y)$
    \end{enumerate}
  \end{multicols}

  \begin{resp}~
    \begin{multicols}{4}
      \begin{enumerate}[a)]
        \item V
        \item F
        \item V
        \item V
      \end{enumerate}
    \end{multicols}
  \end{resp}

  \question  Usando os símbolos predicados e quantificadores, escreva cada declaração como uma fbf predicada. O conjunto universo é o mundo inteiro.

  B(x) é “x é uma bola”.

  R(x) é “x é redonda”.

  S(x) é “x é uma bola de futebol”

  \begin{enumerate}[a)]
    \item Todas as bolas são redondas.
    \item Nem todas as bolas são bolas de futebol.
    \item Todas as bolas de futebol são redondas.
    \item Algumas bolas não são redondas.
    \item Algumas bolas são redondas, mas as bolas de futebol não são.
    \item Toda bola redonda é uma bola de futebol.
    \item Se as bolas de futebol são redondas, então todas as bolas são redondas.
  \end{enumerate}

  \begin{resp}~
    - 
  \end{resp}

  \question  Justifique cada passo na sequência de demonstração a seguir para a fbf.

  \[ (\exists x)[P(x) \rightarrow Q(x)] \rightarrow [(\forall x)P(x) \rightarrow (\exists x)Q(x)] \] 

  \begin{enumerate}[1.]
    \item $(\exists x)[P(x) \rightarrow Q(x)]$
    \item $P(a) \rightarrow Q(a) $
    \item $(\forall x)P(x)$
    \item $P(a)$
    \item $Q(a)$
    \item $(\exists x)Q(x)$
  \end{enumerate}

  \begin{resp}~
    - 
  \end{resp}

  \question Prove que cada fbf a seguir é um argumento válido.

    \begin{enumerate}[a.]
      \item $(\forall x)P(x) \rightarrow (\forall x)[P(x) \lor Q(x)]$
      \item $(\exists x)(\exists y)P(x, y) \rightarrow (\exists y)(\exists x)P(x, y)$
      \item $(\forall x)P(x) \land (\exists x)[P(x)]' \rightarrow (\exists x)Q(x)$
  \end{enumerate}

  \begin{resp}~
    - 
  \end{resp}
  
\end{questions}

\vspace{1cm}
%\begin{center}
%  \section*{Gabarito}
%\end{center}
%\Closesolutionfile{ans}% finaliza a gravação das respostas
%\begin{Gabarito}{1}
~
    -
  
\end{Gabarito}
\begin{Gabarito}{2}
    -
  
\end{Gabarito}
\begin{Gabarito}{3}
    -
    %  \begin{multicols}{4}
    %    \begin{enumerate}[a)]
    %      \item 2
    %      \item 5/3
    %      \item 0
    %      \item 1
    %    \end{enumerate}
    %  \end{multicols}
  
\end{Gabarito}
\begin{Gabarito}{4}
~
    -
  
\end{Gabarito}
\begin{Gabarito}{5}
~
    -
  
\end{Gabarito}
\begin{Gabarito}{6}
~
    -
  
\end{Gabarito}
\begin{Gabarito}{7}
~
    -
  
\end{Gabarito}
\begin{Gabarito}{8}
~
    -
  
\end{Gabarito}
\begin{Gabarito}{9}
~
    -
  
\end{Gabarito}
\begin{Gabarito}{10}
~
    -
  
\end{Gabarito}
\begin{Gabarito}{11}
~
    -
  
\end{Gabarito}
\begin{Gabarito}{12}
~
    -
  
\end{Gabarito}
 %imprime as soluções
\end{document}
